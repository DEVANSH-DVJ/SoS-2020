``Stars are hot balls of burning gases", is the one universal truth about stars each and every student is taught since school. Of course, this statement is absolutely correct, but understanding the mechanism that stars employ to ``burn" these gases, the reasons why these specific mechanisms are employed, and how they are sustained, is a whole different story altogether. \\\\
If you want important insight into how stars work, just go out and look at them for a few nights.. You fill find that they appear to do nothing other than shine steadily. This is certainly true from a historic perspective, taking the Sun as an example,from fossil evidence, we can extend this period of inactivity to nearly five billion years! \\\\
The reason for this relative tranquility is that stars are in general very stable objects in that the self-gravitational forces are delicately balanced by steep internal pressure gradients. The latter require high temperatures. In the deep interior of stars, these temperatures are measured to be of the order of millions of Kelvins, and in most cases are sufficiently high to initiate the thermonuclear fusion of light nuclei. The power so produced then laboriously works its way out of the remaining bulk of the star and finally gives rise to the radiation we see streaming off the surface. The vast majority of stars spend much of their active lives in such an equilibrium state, gradually converting Hydrogen into Helium, and it is only this gradual transmutation of elements by the fusion process that eventually causes the structure to change in some marked way.\\\\

\begin{figure}[H]
    \centering
    \color{white}
    \includegraphics[scale=0.25]{evol.jpg}
    \caption{\color{white}An artist's impression of stellar evolution. Of course, all this shall be elaborated upon later in the article}
    \label{fig:my_label}
\end{figure}
\noindent
In this article, I am going to focus on the mechanism responsible for this equilibrium state, as well as those responsible for any changes in it. I am then going to show how these apply specifically to the Sun, White Dwarfs and certain variable stars too. This article requires the user to have a basic knowledge of thermodynamics, including the theory of ideal gases, as well as Newton's laws of Gravitation. I shall not spend much time on the basics, but delve directly into the mathematics that govern stellar dynamics and evolution. 
