\documentclass[12pt]{article}
\usepackage[utf8]{inputenc}
\usepackage{xcolor}
\usepackage[margin=1in]{geometry}
% \usepackage{graphicx}
\usepackage{float}
% \usepackage{textcomp}
% \usepackage{siunitx}
\usepackage{fancyhdr}
% \usepackage{amsmath, amssymb, amsthm}
\usepackage{tikz}
\usepackage{tikzpagenodes}
% \usepackage{eso-pic}
% \usepackage{enumerate}
\usepackage[colorlinks]{hyperref}
\hypersetup{
    colorlinks = true,
    linkcolor = blue,
    urlcolor = blue,
    pdftitle={Graph Theory and Algorithms}
}
% \usepackage{lipsum}
% \usepackage{etoolbox}
\usepackage{bookmark}
\usepackage{enumitem}

% \makeatletter

% \renewcommand\float@endH{\@endfloatbox\vskip\intextsep
%   \if@flstyle\setbox\@currbox\float@makebox\columnwidth\fi
%   \box\@currbox\vskip\intextsep\relax\@doendpe}

% \patchcmd{\f@nch@head}{\rlap}{\color{white}\rlap}{}{}
% \patchcmd{\headrule}{\hrule}{\color{white}\hrule}{}{}
% \patchcmd{\f@nch@foot}{\rlap}{\color{white}\rlap}{}{}
% \patchcmd{\footrule}{\hrule}{\color{white}\hrule}{}{}

% \makeatother

%%% Header and footer
%%% ------------------------------------------
\pagestyle{fancy}
\fancyhf{}
\rhead{Summer of Science : Graph Theory}
% \fancyhead[L]{\nouppercase{\leftmark}}
% \cfoot{\thepage}

%%% Macros
%%% ------------------------------------------
% \newcommand{\desctotoc}[4]{%
%   \addtocontents{toc}{\medskip\noindent\detokenize{#1}\leavevmode\par\medskip}
% }


\begin{document}


%%% Titlepage
%%% ------------------------------------------
\tikz[overlay] \node[opacity=1,inner sep=0pt] at (7.65,-12.5) {\includegraphics[width=\paperwidth]{./Graph-Theory-768x481.jpg}};
\vspace*{3cm}
\thispagestyle{empty}
	\begin{center}
	\textbf{\Huge{Graph Theory and Algorithms}}\\
	\textbf{\large{\\ An Introduction to the world of Graphs }}
	\end{center}
\vfill
	\begin{center}
	\large{\textbf{Devansh Jain,}\\
    Indian Institute of Technology, Bombay\\
    \vspace{0.5cm}
    (Mentor: Tathagat Verma)\\}
	\end{center}

\newpage

\pagenumbering{roman}
\begin{center}
\hspace{0pt}
    \tableofcontents
    \vfill
\hspace{0pt}
P.S. The page numbers are depicted on the top-left of the image of the hand-written notes\\
\end{center}

\newpage

\section*{Preface}
\addcontentsline{toc}{section}{Preface}
\begin{flushleft}
Preface
\subsubsection*{To the evaluators}
adedsd
\subsubsection*{To the students}
sdsd

\end{flushleft}

\newpage
\pagenumbering{Roman}
\setcounter{page}{1}
{\color{black} \section*{MIT 6.042J/18.062J \small{(Fall 2010) : Mathematics for Computer Science}}}
\addcontentsline{toc}{section}{MIT 6.042J/18.062J (Fall 2010) : Mathematics for Computer Science}
\begin{flushleft}
\noindent
This course is available on OpenCourseWare MIT (ocw.mit.edu) as well as on Youtube channel MIT OpenCourseWare.\\
\subsubsection*{Course Instructor(s)}
Prof. Tom Leighton\\
Dr. Marten van Dijk
\subsubsection*{Course Description}
This course covers elementary discrete mathematics for computer science and engineering. It emphasizes mathematical definitions and proofs as well as applicable methods. Topics include formal logic notation, proof methods; induction, well-ordering; sets, relations; elementary graph theory; integer congruences; asymptotic notation and growth of functions; permutations and combinations, counting principles; discrete probability. Further selected topics may also be covered, such as recursive definition and structural induction; state machines and invariants; recurrences; generating functions.\\
\vspace{1cm}
In this report, I have included notes of Lectures 6 thru 11 which cover\\ 'Mathematical basics of Graph Theory'.\\
More detailed notes of the course are present \href{https://github.com/DEVANSH-DVJ/MIT-6.042J-18.062J-Fall-2010}{here}\\

\end{flushleft}
\newpage
\pagenumbering{arabic}
\setcounter{page}{23}
{\color{black} \subsection*{Lecture 6: Graph Theory and Coloring}}
\addcontentsline{toc}{subsection}{Lecture 6: Graph Theory and Coloring}
\begin{figure}[H]
    \centering
    \includegraphics[width=16cm, height=21cm]{"./MIT-6.042J/MIT-6042J-023"}
\end{figure}

\newpage
{\color{black} \subsection*{Lecture 6: Graph Theory and Coloring}}
% \addcontentsline{toc}{subsection}{Lecture 6: Graph Theory and Coloring}
\begin{figure}[H]
    \centering
    \includegraphics[width=16cm, height=21cm]{"./MIT-6.042J/MIT-6042J-024"}
\end{figure}

\newpage
{\color{black} \subsection*{Lecture 6: Graph Theory and Coloring}}
% \addcontentsline{toc}{subsection}{Lecture 6: Graph Theory and Coloring}
\begin{figure}[H]
    \centering
    \includegraphics[width=16cm, height=21cm]{"./MIT-6.042J/MIT-6042J-025"}
\end{figure}

\newpage
{\color{black} \subsection*{Lecture 6: Graph Theory and Coloring}}
% \addcontentsline{toc}{subsection}{Lecture 6: Graph Theory and Coloring}
\begin{figure}[H]
    \centering
    \includegraphics[width=16cm, height=21cm]{"./MIT-6.042J/MIT-6042J-026"}
\end{figure}

\newpage
{\color{black} \subsection*{Lecture 6: Graph Theory and Coloring}}
% \addcontentsline{toc}{subsection}{Lecture 6: Graph Theory and Coloring}
\begin{figure}[H]
    \centering
    \includegraphics[width=16cm, height=21cm]{"./MIT-6.042J/MIT-6042J-027"}
\end{figure}

\newpage
{\color{black} \subsection*{Lecture 6: Graph Theory and Coloring}}
% \addcontentsline{toc}{subsection}{Lecture 6: Graph Theory and Coloring}
\begin{figure}[H]
    \centering
    \includegraphics[width=16cm, height=21cm]{"./MIT-6.042J/MIT-6042J-028"}
\end{figure}

\newpage
{\color{black} \subsection*{Lecture 6: Graph Theory and Coloring}}
% \addcontentsline{toc}{subsection}{Lecture 6: Graph Theory and Coloring}
\begin{figure}[H]
    \centering
    \includegraphics[width=16cm, height=21cm]{"./MIT-6.042J/MIT-6042J-029"}
\end{figure}

\newpage
{\color{black} \subsection*{Lecture 7: Matching Problems}}
\addcontentsline{toc}{subsection}{Lecture 7: Matching Problems}
\begin{figure}[H]
    \centering
    \includegraphics[width=16cm, height=21cm]{"./MIT-6.042J/MIT-6042J-030"}
\end{figure}

\newpage
{\color{black} \subsection*{Lecture 7: Matching Problems}}
% \addcontentsline{toc}{subsection}{Lecture 7: Matching Problems}
\begin{figure}[H]
    \centering
    \includegraphics[width=16cm, height=21cm]{"./MIT-6.042J/MIT-6042J-031"}
\end{figure}

\newpage
{\color{black} \subsection*{Lecture 7: Matching Problems}}
% \addcontentsline{toc}{subsection}{Lecture 7: Matching Problems}
\begin{figure}[H]
    \centering
    \includegraphics[width=16cm, height=21cm]{"./MIT-6.042J/MIT-6042J-032"}
\end{figure}

\newpage
{\color{black} \subsection*{Lecture 7: Matching Problems}}
% \addcontentsline{toc}{subsection}{Lecture 7: Matching Problems}
\begin{figure}[H]
    \centering
    \includegraphics[width=16cm, height=21cm]{"./MIT-6.042J/MIT-6042J-033"}
\end{figure}

\newpage
{\color{black} \subsection*{Lecture 7: Matching Problems}}
% \addcontentsline{toc}{subsection}{Lecture 7: Matching Problems}
\begin{figure}[H]
    \centering
    \includegraphics[width=16cm, height=21cm]{"./MIT-6.042J/MIT-6042J-034"}
\end{figure}

\newpage
{\color{black} \subsection*{Lecture 7: Matching Problems}}
% \addcontentsline{toc}{subsection}{Lecture 7: Matching Problems}
\begin{figure}[H]
    \centering
    \includegraphics[width=16cm, height=21cm]{"./MIT-6.042J/MIT-6042J-035"}
\end{figure}

\newpage
{\color{black} \subsection*{Lecture 7: Matching Problems}}
% \addcontentsline{toc}{subsection}{Lecture 7: Matching Problems}
\begin{figure}[H]
    \centering
    \includegraphics[width=16cm, height=21cm]{"./MIT-6.042J/MIT-6042J-036"}
\end{figure}

\newpage
{\color{black} \subsection*{Lecture 7: Matching Problems}}
% \addcontentsline{toc}{subsection}{Lecture 7: Matching Problems}
\begin{figure}[H]
    \centering
    \includegraphics[width=16cm, height=21cm]{"./MIT-6.042J/MIT-6042J-037"}
\end{figure}

\newpage
{\color{black} \subsection*{Lecture 8: Graph Theory II: Minimum Spanning Trees}}
\addcontentsline{toc}{subsection}{Lecture 8: Graph Theory II: Minimum Spanning Trees}
\begin{figure}[H]
    \centering
    \includegraphics[width=16cm, height=21cm]{"./MIT-6.042J/MIT-6042J-038"}
\end{figure}

\newpage
{\color{black} \subsection*{Lecture 8: Graph Theory II: Minimum Spanning Trees}}
% \addcontentsline{toc}{subsection}{Lecture 8: Graph Theory II: Minimum Spanning Trees}
\begin{figure}[H]
    \centering
    \includegraphics[width=16cm, height=21cm]{"./MIT-6.042J/MIT-6042J-039"}
\end{figure}

\newpage
{\color{black} \subsection*{Lecture 8: Graph Theory II: Minimum Spanning Trees}}
% \addcontentsline{toc}{subsection}{Lecture 8: Graph Theory II: Minimum Spanning Trees}
\begin{figure}[H]
    \centering
    \includegraphics[width=16cm, height=21cm]{"./MIT-6.042J/MIT-6042J-040"}
\end{figure}

\newpage
{\color{black} \subsection*{Lecture 8: Graph Theory II: Minimum Spanning Trees}}
% \addcontentsline{toc}{subsection}{Lecture 8: Graph Theory II: Minimum Spanning Trees}
\begin{figure}[H]
    \centering
    \includegraphics[width=16cm, height=21cm]{"./MIT-6.042J/MIT-6042J-041"}
\end{figure}

\newpage
{\color{black} \subsection*{Lecture 8: Graph Theory II: Minimum Spanning Trees}}
% \addcontentsline{toc}{subsection}{Lecture 8: Graph Theory II: Minimum Spanning Trees}
\begin{figure}[H]
    \centering
    \includegraphics[width=16cm, height=21cm]{"./MIT-6.042J/MIT-6042J-042"}
\end{figure}

\newpage
{\color{black} \subsection*{Lecture 8: Graph Theory II: Minimum Spanning Trees}}
% \addcontentsline{toc}{subsection}{Lecture 8: Graph Theory II: Minimum Spanning Trees}
\begin{figure}[H]
    \centering
    \includegraphics[width=16cm, height=21cm]{"./MIT-6.042J/MIT-6042J-043"}
\end{figure}

\newpage
{\color{black} \subsection*{Lecture 8: Graph Theory II: Minimum Spanning Trees}}
% \addcontentsline{toc}{subsection}{Lecture 8: Graph Theory II: Minimum Spanning Trees}
\begin{figure}[H]
    \centering
    \includegraphics[width=16cm, height=21cm]{"./MIT-6.042J/MIT-6042J-044"}
\end{figure}

\newpage
{\color{black} \subsection*{Lecture 8: Graph Theory II: Minimum Spanning Trees}}
% \addcontentsline{toc}{subsection}{Lecture 8: Graph Theory II: Minimum Spanning Trees}
\begin{figure}[H]
    \centering
    \includegraphics[width=16cm, height=21cm]{"./MIT-6.042J/MIT-6042J-045"}
\end{figure}

\newpage
{\color{black} \subsection*{Lecture 8: Graph Theory II: Minimum Spanning Trees}}
% \addcontentsline{toc}{subsection}{Lecture 8: Graph Theory II: Minimum Spanning Trees}
\begin{figure}[H]
    \centering
    \includegraphics[width=16cm, height=21cm]{"./MIT-6.042J/MIT-6042J-046"}
\end{figure}

\newpage
{\color{black} \subsection*{Lecture 8: Graph Theory II: Minimum Spanning Trees}}
% \addcontentsline{toc}{subsection}{Lecture 8: Graph Theory II: Minimum Spanning Trees}
\begin{figure}[H]
    \centering
    \includegraphics[width=16cm, height=21cm]{"./MIT-6.042J/MIT-6042J-047"}
\end{figure}

\newpage
{\color{black} \subsection*{Lecture 8: Graph Theory II: Minimum Spanning Trees}}
% \addcontentsline{toc}{subsection}{Lecture 8: Graph Theory II: Minimum Spanning Trees}
\begin{figure}[H]
    \centering
    \includegraphics[width=16cm, height=21cm]{"./MIT-6.042J/MIT-6042J-048"}
\end{figure}

\newpage
{\color{black} \subsection*{Lecture 9: Communication Networks}}
\addcontentsline{toc}{subsection}{Lecture 9: Communication Networks}
\begin{figure}[H]
    \centering
    \includegraphics[width=16cm, height=21cm]{"./MIT-6.042J/MIT-6042J-049"}
\end{figure}

\newpage
{\color{black} \subsection*{Lecture 9: Communication Networks}}
% \addcontentsline{toc}{subsection}{Lecture 9: Communication Networks}
\begin{figure}[H]
    \centering
    \includegraphics[width=16cm, height=21cm]{"./MIT-6.042J/MIT-6042J-050"}
\end{figure}

\newpage
{\color{black} \subsection*{Lecture 9: Communication Networks}}
% \addcontentsline{toc}{subsection}{Lecture 9: Communication Networks}
\begin{figure}[H]
    \centering
    \includegraphics[width=16cm, height=21cm]{"./MIT-6.042J/MIT-6042J-051"}
\end{figure}

\newpage
{\color{black} \subsection*{Lecture 9: Communication Networks}}
% \addcontentsline{toc}{subsection}{Lecture 9: Communication Networks}
\begin{figure}[H]
    \centering
    \includegraphics[width=16cm, height=21cm]{"./MIT-6.042J/MIT-6042J-052"}
\end{figure}

\newpage
{\color{black} \subsection*{Lecture 9: Communication Networks}}
% \addcontentsline{toc}{subsection}{Lecture 9: Communication Networks}
\begin{figure}[H]
    \centering
    \includegraphics[width=16cm, height=21cm]{"./MIT-6.042J/MIT-6042J-053"}
\end{figure}

\newpage
{\color{black} \subsection*{Lecture 9: Communication Networks}}
% \addcontentsline{toc}{subsection}{Lecture 9: Communication Networks}
\begin{figure}[H]
    \centering
    \includegraphics[width=16cm, height=21cm]{"./MIT-6.042J/MIT-6042J-054"}
\end{figure}

\newpage
{\color{black} \subsection*{Lecture 9: Communication Networks}}
% \addcontentsline{toc}{subsection}{Lecture 9: Communication Networks}
\begin{figure}[H]
    \centering
    \includegraphics[width=16cm, height=21cm]{"./MIT-6.042J/MIT-6042J-055"}
\end{figure}

\newpage
{\color{black} \subsection*{Lecture 9: Communication Networks}}
% \addcontentsline{toc}{subsection}{Lecture 9: Communication Networks}
\begin{figure}[H]
    \centering
    \includegraphics[width=16cm, height=21cm]{"./MIT-6.042J/MIT-6042J-056"}
\end{figure}

\newpage
{\color{black} \subsection*{Lecture 10: Graph Theory III}}
\addcontentsline{toc}{subsection}{Lecture 10: Graph Theory III}
\begin{figure}[H]
    \centering
    \includegraphics[width=16cm, height=21cm]{"./MIT-6.042J/MIT-6042J-057"}
\end{figure}

\newpage
{\color{black} \subsection*{Lecture 10: Graph Theory III}}
% \addcontentsline{toc}{subsection}{Lecture 10: Graph Theory III}
\begin{figure}[H]
    \centering
    \includegraphics[width=16cm, height=21cm]{"./MIT-6.042J/MIT-6042J-058"}
\end{figure}

\newpage
{\color{black} \subsection*{Lecture 10: Graph Theory III}}
% \addcontentsline{toc}{subsection}{Lecture 10: Graph Theory III}
\begin{figure}[H]
    \centering
    \includegraphics[width=16cm, height=21cm]{"./MIT-6.042J/MIT-6042J-059"}
\end{figure}

\newpage
{\color{black} \subsection*{Lecture 10: Graph Theory III}}
% \addcontentsline{toc}{subsection}{Lecture 10: Graph Theory III}
\begin{figure}[H]
    \centering
    \includegraphics[width=16cm, height=21cm]{"./MIT-6.042J/MIT-6042J-060"}
\end{figure}

\newpage
{\color{black} \subsection*{Lecture 10: Graph Theory III}}
% \addcontentsline{toc}{subsection}{Lecture 10: Graph Theory III}
\begin{figure}[H]
    \centering
    \includegraphics[width=16cm, height=21cm]{"./MIT-6.042J/MIT-6042J-061"}
\end{figure}

\newpage
{\color{black} \subsection*{Lecture 10: Graph Theory III}}
% \addcontentsline{toc}{subsection}{Lecture 10: Graph Theory III}
\begin{figure}[H]
    \centering
    \includegraphics[width=16cm, height=21cm]{"./MIT-6.042J/MIT-6042J-062"}
\end{figure}

\newpage
{\color{black} \subsection*{Lecture 10: Graph Theory III}}
% \addcontentsline{toc}{subsection}{Lecture 10: Graph Theory III}
\begin{figure}[H]
    \centering
    \includegraphics[width=16cm, height=21cm]{"./MIT-6.042J/MIT-6042J-063"}
\end{figure}

\newpage
{\color{black} \subsection*{Lecture 10: Graph Theory III}}
% \addcontentsline{toc}{subsection}{Lecture 10: Graph Theory III}
\begin{figure}[H]
    \centering
    \includegraphics[width=16cm, height=21cm]{"./MIT-6.042J/MIT-6042J-064"}
\end{figure}

\newpage
{\color{black} \subsection*{Lecture 11: Relations, Partial Orders, and Scheduling}}
\addcontentsline{toc}{subsection}{Lecture 11: Relations, Partial Orders, and Scheduling}
\begin{figure}[H]
    \centering
    \includegraphics[width=16cm, height=21cm]{"./MIT-6.042J/MIT-6042J-065"}
\end{figure}

\newpage
{\color{black} \subsection*{Lecture 11: Relations, Partial Orders, and Scheduling}}
% \addcontentsline{toc}{subsection}{Lecture 11: Relations, Partial Orders, and Scheduling}
\begin{figure}[H]
    \centering
    \includegraphics[width=16cm, height=21cm]{"./MIT-6.042J/MIT-6042J-066"}
\end{figure}

\newpage
{\color{black} \subsection*{Lecture 11: Relations, Partial Orders, and Scheduling}}
% \addcontentsline{toc}{subsection}{Lecture 11: Relations, Partial Orders, and Scheduling}
\begin{figure}[H]
    \centering
    \includegraphics[width=16cm, height=21cm]{"./MIT-6.042J/MIT-6042J-067"}
\end{figure}

\newpage
{\color{black} \subsection*{Lecture 11: Relations, Partial Orders, and Scheduling}}
% \addcontentsline{toc}{subsection}{Lecture 11: Relations, Partial Orders, and Scheduling}
\begin{figure}[H]
    \centering
    \includegraphics[width=16cm, height=21cm]{"./MIT-6.042J/MIT-6042J-068"}
\end{figure}

\newpage
{\color{black} \subsection*{Lecture 11: Relations, Partial Orders, and Scheduling}}
% \addcontentsline{toc}{subsection}{Lecture 11: Relations, Partial Orders, and Scheduling}
\begin{figure}[H]
    \centering
    \includegraphics[width=16cm, height=21cm]{"./MIT-6.042J/MIT-6042J-069"}
\end{figure}

\newpage
{\color{black} \subsection*{Lecture 11: Relations, Partial Orders, and Scheduling}}
% \addcontentsline{toc}{subsection}{Lecture 11: Relations, Partial Orders, and Scheduling}
\begin{figure}[H]
    \centering
    \includegraphics[width=16cm, height=21cm]{"./MIT-6.042J/MIT-6042J-070"}
\end{figure}

\newpage
{\color{black} \subsection*{Lecture 11: Relations, Partial Orders, and Scheduling}}
% \addcontentsline{toc}{subsection}{Lecture 11: Relations, Partial Orders, and Scheduling}
\begin{figure}[H]
    \centering
    \includegraphics[width=16cm, height=21cm]{"./MIT-6.042J/MIT-6042J-071"}
\end{figure}

\newpage
{\color{black} \subsection*{Lecture 11: Relations, Partial Orders, and Scheduling}}
% \addcontentsline{toc}{subsection}{Lecture 11: Relations, Partial Orders, and Scheduling}
\begin{figure}[H]
    \centering
    \includegraphics[width=16cm, height=21cm]{"./MIT-6.042J/MIT-6042J-072"}
\end{figure}

\newpage
{\color{black} \subsection*{Part of Lecture 13: Asymptotics}}
\addcontentsline{toc}{subsection}{Part of Lecture 13: Asymptotics}
\begin{figure}[H]
    \centering
    \includegraphics[width=16cm, height=21cm]{"./MIT-6.042J/MIT-6042J-073"}
\end{figure}

\newpage
{\color{black} \subsection*{Part of Lecture 13: Asymptotics}}
% \addcontentsline{toc}{subsection}{Part of Lecture 13: Asymptotics}
\begin{figure}[H]
    \centering
    \includegraphics[width=16cm, height=21cm]{"./MIT-6.042J/MIT-6042J-074"}
\end{figure}

\newpage
{\color{black} \subsection*{Part of Lecture 13: Asymptotics}}
% \addcontentsline{toc}{subsection}{Part of Lecture 13: Asymptotics}
\begin{figure}[H]
    \centering
    \includegraphics[width=16cm, height=21cm]{"./MIT-6.042J/MIT-6042J-075"}
\end{figure}

\newpage
{\color{black} \subsection*{Part of Lecture 13: Asymptotics}}
% \addcontentsline{toc}{subsection}{Part of Lecture 13: Asymptotics}
\begin{figure}[H]
    \centering
    \includegraphics[width=16cm, height=21cm]{"./MIT-6.042J/MIT-6042J-076"}
\end{figure}

\newpage
{\color{black} \subsection*{Part of Lecture 13: Asymptotics}}
% \addcontentsline{toc}{subsection}{Part of Lecture 13: Asymptotics}
\begin{figure}[H]
    \centering
    \includegraphics[width=16cm, height=21cm]{"./MIT-6.042J/MIT-6042J-077"}
\end{figure}


\newpage
\pagenumbering{Roman}
\setcounter{page}{2}
{\color{black} \section*{MIT 6.006 \small{(Fall 2011) : Introduction to Algorithms}}}
\addcontentsline{toc}{section}{MIT 6.006 (Fall 2011) : Introduction to Algorithms}
\begin{flushleft}
\noindent
This course is available on OpenCourseWare MIT (ocw.mit.edu) as well as on Youtube channel MIT OpenCourseWare.\\
\subsubsection*{Course Instructor(s)}
Prof. Erik Demaine\\
Prof. Srini Devadas
\subsubsection*{Course Description}
This course provides an introduction to mathematical modeling of computational problems. It covers the common algorithms, algorithmic paradigms, and data structures used to solve these problems. The course emphasizes the relationship between algorithms and programming, and introduces basic performance measures and analysis techniques for these problems.\\
\vspace{1cm}
In this report, I have included my notes of Lectures 13 thru 18 which cover\\ 'Basic algorithms of Graph Theory'.\\
More detailed notes of the course are present \href{https://github.com/devansh-dvj/MIT-6.006-Fall-2011}{here}\\

\end{flushleft}
\newpage
\pagenumbering{arabic}
\setcounter{page}{95}
\newpage
\begin{figure}[H]
    \centering
    \includegraphics[scale=0.25]{"./MIT 6.006/MIT_6006_095"}
\end{figure}
\newpage
\begin{figure}[H]
    \centering
    \includegraphics[scale=0.25]{"./MIT 6.006/MIT_6006_096"}
\end{figure}
\newpage
\begin{figure}[H]
    \centering
    \includegraphics[scale=0.25]{"./MIT 6.006/MIT_6006_097"}
\end{figure}
\newpage
\begin{figure}[H]
    \centering
    \includegraphics[scale=0.25]{"./MIT 6.006/MIT_6006_098"}
\end{figure}
\newpage
\begin{figure}[H]
    \centering
    \includegraphics[scale=0.25]{"./MIT 6.006/MIT_6006_099"}
\end{figure}
\newpage
\begin{figure}[H]
    \centering
    \includegraphics[scale=0.25]{"./MIT 6.006/MIT_6006_100"}
\end{figure}
\newpage
\begin{figure}[H]
    \centering
    \includegraphics[scale=0.25]{"./MIT 6.006/MIT_6006_101"}
\end{figure}
\newpage
\begin{figure}[H]
    \centering
    \includegraphics[scale=0.25]{"./MIT 6.006/MIT_6006_102"}
\end{figure}
\newpage
\begin{figure}[H]
    \centering
    \includegraphics[scale=0.25]{"./MIT 6.006/MIT_6006_103"}
\end{figure}
\newpage
\begin{figure}[H]
    \centering
    \includegraphics[scale=0.25]{"./MIT 6.006/MIT_6006_104"}
\end{figure}
\newpage
\begin{figure}[H]
    \centering
    \includegraphics[scale=0.25]{"./MIT 6.006/MIT_6006_105"}
\end{figure}
\newpage
\begin{figure}[H]
    \centering
    \includegraphics[scale=0.25]{"./MIT 6.006/MIT_6006_106"}
\end{figure}
\newpage
\begin{figure}[H]
    \centering
    \includegraphics[scale=0.25]{"./MIT 6.006/MIT_6006_107"}
\end{figure}
\newpage
\begin{figure}[H]
    \centering
    \includegraphics[scale=0.25]{"./MIT 6.006/MIT_6006_108"}
\end{figure}
\newpage
\begin{figure}[H]
    \centering
    \includegraphics[scale=0.25]{"./MIT 6.006/MIT_6006_109"}
\end{figure}
\newpage
\begin{figure}[H]
    \centering
    \includegraphics[scale=0.25]{"./MIT 6.006/MIT_6006_110"}
\end{figure}
\newpage
\begin{figure}[H]
    \centering
    \includegraphics[scale=0.25]{"./MIT 6.006/MIT_6006_111"}
\end{figure}
\newpage
\begin{figure}[H]
    \centering
    \includegraphics[scale=0.25]{"./MIT 6.006/MIT_6006_112"}
\end{figure}
\newpage
\begin{figure}[H]
    \centering
    \includegraphics[scale=0.25]{"./MIT 6.006/MIT_6006_113"}
\end{figure}
\newpage
\begin{figure}[H]
    \centering
    \includegraphics[scale=0.25]{"./MIT 6.006/MIT_6006_114"}
\end{figure}
\newpage
\begin{figure}[H]
    \centering
    \includegraphics[scale=0.25]{"./MIT 6.006/MIT_6006_115"}
\end{figure}
\newpage
\begin{figure}[H]
    \centering
    \includegraphics[scale=0.25]{"./MIT 6.006/MIT_6006_116"}
\end{figure}
\newpage
\begin{figure}[H]
    \centering
    \includegraphics[scale=0.25]{"./MIT 6.006/MIT_6006_117"}
\end{figure}
\newpage
\begin{figure}[H]
    \centering
    \includegraphics[scale=0.25]{"./MIT 6.006/MIT_6006_118"}
\end{figure}
\newpage
\begin{figure}[H]
    \centering
    \includegraphics[scale=0.25]{"./MIT 6.006/MIT_6006_119"}
\end{figure}
\newpage
\begin{figure}[H]
    \centering
    \includegraphics[scale=0.25]{"./MIT 6.006/MIT_6006_120"}
\end{figure}
\newpage
\begin{figure}[H]
    \centering
    \includegraphics[scale=0.25]{"./MIT 6.006/MIT_6006_121"}
\end{figure}
\newpage
\begin{figure}[H]
    \centering
    \includegraphics[scale=0.25]{"./MIT 6.006/MIT_6006_122"}
\end{figure}
\newpage
\begin{figure}[H]
    \centering
    \includegraphics[scale=0.25]{"./MIT 6.006/MIT_6006_123"}
\end{figure}
\newpage
\begin{figure}[H]
    \centering
    \includegraphics[scale=0.25]{"./MIT 6.006/MIT_6006_124"}
\end{figure}
\newpage
\begin{figure}[H]
    \centering
    \includegraphics[scale=0.25]{"./MIT 6.006/MIT_6006_125"}
\end{figure}
\newpage
\begin{figure}[H]
    \centering
    \includegraphics[scale=0.25]{"./MIT 6.006/MIT_6006_126"}
\end{figure}
\newpage
\begin{figure}[H]
    \centering
    \includegraphics[scale=0.25]{"./MIT 6.006/MIT_6006_127"}
\end{figure}
\newpage
\begin{figure}[H]
    \centering
    \includegraphics[scale=0.25]{"./MIT 6.006/MIT_6006_128"}
\end{figure}
\newpage
\begin{figure}[H]
    \centering
    \includegraphics[scale=0.25]{"./MIT 6.006/MIT_6006_129"}
\end{figure}
\newpage
\begin{figure}[H]
    \centering
    \includegraphics[scale=0.25]{"./MIT 6.006/MIT_6006_130"}
\end{figure}
\newpage
\begin{figure}[H]
    \centering
    \includegraphics[scale=0.25]{"./MIT 6.006/MIT_6006_131"}
\end{figure}
\newpage
\begin{figure}[H]
    \centering
    \includegraphics[scale=0.25]{"./MIT 6.006/MIT_6006_132"}
\end{figure}
\newpage
\begin{figure}[H]
    \centering
    \includegraphics[scale=0.25]{"./MIT 6.006/MIT_6006_133"}
\end{figure}


\newpage
\pagenumbering{Roman}
\setcounter{page}{3}
{\color{black} \section*{References}}
\addcontentsline{toc}{section}{References}
\begin{flushleft}
\subsubsection*{Site References:}
\vspace{-2mm}
\begin{itemize}[itemsep = -1 mm]
  \item \url{https://www.geeksforgeeks.org/fundamentals-of-algorithms/}\\(For Basics of algorithms)
  \item \url{https://www.geeksforgeeks.org/graph-data-structure-and-algorithms/}\\(For concepts of graph algorithms)
  \item \url{http://cp-algorithms.com/}\\(Problems on graphs algorithms)
  \item \url{https://www.youtube.com/watch?v=09_LlHjoEiY&t=7s}\\(Summary of graph algorithms)
\end{itemize}

\subsubsection*{Book References:}
\vspace{-2mm}
\begin{itemize}[itemsep = 0 mm]
  \item Introduction to Algorithms - Thomas H. Cormen, Charles E. Leiserson, Ronald L. Rivest, Clifford Stein - Third Edition (Textbook for MIT 6.006)
  \item Introduction to Graph Theory (2nd Edition) - Douglas B. West (For Mathematical concepts of Graph Theory)
  \item GRAPH THEORY - Keijo Ruohonen (For Mathematical concepts of Graph Theory)
  \item Algorithm Design: Parallel and Sequential - Xiuquan Lv (Algorithm Design)
  \item Competitive Programmer’s Handbook - Antti Laaksonen (For competitive programming)
  \item NUS CS3233 - Competitive Programming (For competitive programming)
\end{itemize}

\subsubsection*{MIT Courses (which are related):}
\vspace{-2mm}
\begin{itemize}[itemsep = -1 mm]
  \item \href{https://ocw.mit.edu/courses/electrical-engineering-and-computer-science/6-042j-mathematics-for-computer-science-fall-2010/}{MIT 6.042J/18.062J - Mathematics for Computer science}
  \item \href{https://ocw.mit.edu/courses/electrical-engineering-and-computer-science/6-006-introduction-to-algorithms-fall-2011/}{MIT 6.006 - Introduction to Algorithms}
  \item \href{https://ocw.mit.edu/courses/electrical-engineering-and-computer-science/6-046j-design-and-analysis-of-algorithms-spring-2015/}{MIT 6.046J/18.410J - Design and Analysis of Algorithms}
  \item \href{https://ocw.mit.edu/courses/electrical-engineering-and-computer-science/6-854j-advanced-algorithms-fall-2008/index.htm}{MIT 6.854J/18.415J - Advanced Algorithms}
  \item \href{https://ocw.mit.edu/courses/electrical-engineering-and-computer-science/6-045j-automata-computability-and-complexity-spring-2011/}{MIT 6.045J/18.400J - Automata, Computability, and Complexity}
  \item \href{https://ocw.mit.edu/courses/mathematics/18-404j-theory-of-computation-fall-2006/}{MIT 18.404J/6.840J - Theory of Computation}
  \item \href{https://ocw.mit.edu/courses/electrical-engineering-and-computer-science/6-079-introduction-to-convex-optimization-fall-2009/}{MIT 6.079/6.975 - Introduction to Convex Optimization}
  \item \href{https://ocw.mit.edu/courses/mathematics/18-315-combinatorial-theory-introduction-to-graph-theory-extremal-and-enumerative-combinatorics-spring-2005/}{MIT 18.315 - Combinatorial Theory: Introduction to Graph Theory, Extremal and Enumerative Combinatorics}
\end{itemize}

\noindent
I would be studying these courses as well and my notes I make would be available on \href{https://github.com/devansh-dvj}{Github (ID: devansh-dvj)}

\end{flushleft}

\end{document}
